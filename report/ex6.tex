%%%%%%%%%%%%%%%% Including packages %%%%%%%%%%%%%%%%%
\documentclass[english,12pt]{article}
\usepackage{amsmath,epsfig}
\usepackage[latin1]{inputenc}
\usepackage{calc, epsfig, rotating, amsmath}
\usepackage[T1]{fontenc}
\usepackage{babel}
\usepackage{textcomp} 
\usepackage{epstopdf}
\usepackage{makeidx}
\usepackage{sidecap}
\usepackage{amssymb}
\usepackage{blkarray}
\usepackage{multirow}
\usepackage{float}
\usepackage{ae} %does also load the fontenc package with T1 option
\usepackage{babel}
\usepackage{ifthen,tikz,xkeyval}
\usepackage{todonotes}
\usepackage{fancyhdr, rotating}
\usepackage{cite}
\usepackage{subfigure}
\usepackage{setspace}
\usepackage{xcolor}
\usepackage[intoc]{nomencl}
\usepackage{braket}
\usepackage{appendix}
\usepackage{lscape}
\usepackage[section]{placeins}

%%%%%%%%%%%%%%%%% Title page %%%%%%%%%%%%%%%%%%%%%%%%%
\title{\textbf{TMA4280 - Exercise 6}}
\author{Rolf H. Myhre and Eirik Hjerten�s}
\date{04.04.14}


%%%%%%%%%%%%%%%%% Paragraph settings %%%%%%%%%%%%%%%%%
\setlength{\parindent}{0pt}
\setlength{\parskip}{1ex plus 0.5 ex minus 0.2ex}

%%%%%%%%%%%%%%%% Beginning document %%%%%%%%%%%%%%%%%%

\begin{document}
\bibliographystyle{jcp}

%%%%%%%%%%%%%%%% Redefining commands %%%%%%%%%%%%%%%%%
\renewcommand{\thesubsection}{\alph{subsection}}
\newcommand{\ra}{\ensuremath \rightarrow}

\makeatletter
\renewcommand{\subsection}{\@startsection{subsection}{1}{0mm}{0.5\baselineskip}{0.5 \baselineskip}{\normalfont\normalsize\textbf}}
\renewcommand{\theenumi}{\roman{enumi}}
\renewcommand{\labelenumi}{\theenumi)}
\makeatother

\begin{titlepage}

	\clearpage\maketitle\thispagestyle{empty}
%%%%%%%%%%%%%%%%%%%%%% Main part %%%%%%%%%%%%%%%%%%%%%

%%%%%%%%%%%%%%% Teaching goals %%%%%%%%%%%%%%%%%%%%%%%
\vspace{1.5cm}
\setcounter{tocdepth}{1}
\setcounter{secnumdepth}{0}
\setcounter{secnumdepth}{2}

\begin{center}
\section*{Summary}
In this report, a parallell implementation of a Discrete Sine Transform solver of the two-dimensional Poisson problem is presented. The implementation is a hybrid between MPI and OpenMP suited for a distributed memory system with multi-threading per MPI-process. The developed code is evaluated with respect to accuracy and efficiency. The hybrid version of the program is also compared to a purely MPI code.
\end{center}

\end{titlepage}
\newpage

\section{Introduction}
In this exercise we have studied the two dimensional Poisson problem on a unit square. The problem can be formulated as follows:
	\begin{align}
	-\nabla^2 u = f & \quad \text{in} \quad \Omega = (0,1)\times(0,1)\\
	u = 0 &\quad \text{on} \quad \partial  \Omega
	\end{align}

where $u$ is the solution to the problem and $f$ is a known function. The last equation gives the boundary conditions. We discretize the problem and solve it on a mesh with identical, uniform spacing in both directions. The Laplace operator is discretized using a 5-point stencil. The problem can then be rewritten as a set of coupled linear equations, represented by:
\begin{equation}
\bar{A} \bar{x} = \bar{b}
\end{equation}

In two dimensions, the 5-point stencil results in a pentadiagonal matrix with a bandwith $n$, the number of points along one direction in the mesh. The matrix $\bar{A}$ is still positive definite and symmetric so we know that the problem is solvable and that the solution is unique. The problem can be solved in a number of ways, using different types of iterative or direct methods. Here we 

\section{Results}


\section{Discussion and Conclusions}

\subsection{Accuracy}

\subsection{Speed-up and efficiency}


\bibliography{master}

\end{document}
